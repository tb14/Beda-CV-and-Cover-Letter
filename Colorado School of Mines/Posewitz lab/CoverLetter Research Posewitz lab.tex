\documentclass[11pt,a4paper,sans]{moderncv}

\usepackage{float}
\usepackage{caption}
\usepackage{fancyhdr}
\usepackage{setspace}
\usepackage{verbatim}
\usepackage[scale=0.8, top = 1.8cm, bottom = 3cm]{geometry}

\moderncvstyle{casual}                             
\moderncvcolor{blue}                               


%%%% FIX THIS SO IT LOOKS NICER
\fancypagestyle{titlepage}{
\fancyhead{}
\fancyfoot[CE, CO]{\textcolor{gray}{Via Ramarini 32, Monterotondo RM, 00015 \thinspace\thinspace $\bullet$ \thinspace\thinspace\mobilephonesymbol +39~(388)~059~1507 \thinspace\thinspace $\bullet$ \thinspace\thinspace \faSkype \thinspace\thinspace \href{skype:rabea.jesser}{rabea.jesser} \thinspace\thinspace \thinspace\thinspace \emailsymbol \href{mailto:rabeajesser@web.de}{rabeajesser@web.de} \thinspace\thinspace $\bullet$ \thinspace\thinspace \faLinkedin \thinspace\thinspace \href{https://www.linkedin.com/in/rabeajesser/}{https://www.linkedin.com/in/rabeajesser/}}}}


\name{\vspace{-5mm}Rabea Jesser}{\vspace{3mm}}

\begin{document}

%%%%%%%%%%%%%%%%%%%RESEARCH FOCUS%%%%%%%%%%%%%
(a) hydrogenase enzymes and the production of hydrogen from phototrophic micro-organisms\\
(b) starch and lipid metabolisms in algae, \\
(c) ‘omics’ based approaches applied to defining whole cell metabolic and regulatory pathways, \\
(d) the diversity of water-oxidizing phototrophs that are adapted to saline ecosystems, and \\
(e) the enzymatic control of metabolic flux in algae. \\
Our research is firmly entrenched in developing a more informed understanding of central metabolism in these fascinating organisms, which can hopefully be applied in viable bioenergy technologies.\\


\thispagestyle{titlepage}
\recipient{Colorado School of Mines}{1500 Illinois St., Golden, CO 80401}
%%%% DONT FORGET TO UPDATE THIS OR USE \today
\date{\today}
%%%%% IF YOU KNOW WHO YOU'RE SENDING THIS TO THEN GIVE THEIR NAME
\opening{Dear Dr. Matthew C. Posewitz,}
\closing{Sincerely,}
%\enclosure[Note]{Reference contacts and/or letters of reference will be sent upon request.}          
\makelettertitle
\begin{spacing}{1.1}

Given my hands-on molecular biology experience, a strong work ethic and precise attention to detail, I would make an excellent Research Associate for the Avery lab and relish the opportunity to help diagnosing companion animals around the US and the world. 
After a B.Sc. in Biology and participation in the iGEM (international Genetically Engineered Machine) competition, I received a M.Sc. in Biology before starting to work with Europe's flagship laboratory for the life sciences, EMBL, in Rome. 
Across various research projects I have worked both independently and as part of a team, gaining strong problem-solving, organizational and communication skills and a proven ability to quickly adapt to new challenges.
\par%\vspace*{1mm}

My theses {\href{https://www.ncbi.nlm.nih.gov/pubmed/29517395}{\textcolor{blue}{{\textit{(Publication)}}}}} have left me with excellent critical thinking skills and well-versed in gathering, analyzing, interpreting and communicating scientific data with meticulous attention to detail. 
More specifically, I have substantial experience in using PCR for molecular biology purposes and, with a proven ability to adapt to new topics and environments, I will have no issue familiarizing myself with flow cytometry assays. 
As an example, during a reorganization of my current lab, I consulted relevant literature to establish standard protocols for the production of viral vector tools in mammalian cell culture and their quantification via qPCR. 
Here, I precisely documented and monitored the generated data to identify areas for improvement and confered with my supervisor to continually improve the procedure. \par%\vspace*{0.5mm}
For the iGEM competition I participated in an interdisciplinary team of 20 highly motivated students.
Within a month we agreed on an influential topic, and after only eight months of wet lab we produced high-value  {\href{https://www.ncbi.nlm.nih.gov/pubmed/29803867}{\textcolor{blue}{\textit{data}}}}, finally presenting our team's findings on our {\href{http://2015.igem.org/Team:Freiburg}{\textcolor{blue}{\textit{website}}}} and at an international conference. 
This experience has not only left me with a strong work ethic, but also with the ability to efficiently identify the essentials from scientific findings and to present these in a comprehensive way to veterinarians or other clients. 
During this time we successfully ran our own lab independently, including ordering reagents, communicating with companies and funding agencies for financial or material aid, making and sterilizing media and keeping the lab and our data organized. \par 
At my current position our facility only consists of my supervisor and me, meaning I not only produce high-quality viral vectors and genetic constructs independently but also maintain a functional laboratory. 
Furthermore, I took the initiative to improve the facility's database (FileMaker)---leaving me with excellent organizational skills I can invest to ensure accurate data recording and storage at the Clinical Immunology Laboratory.  
I have also attended a professional course on presenting with impact and project management for scientists and am able to apply these techniques to balance multiple projects---currently more than 12 at once---while managing my time to deliver final products as quickly as possible.  
Additionally, as part of a {\href{https://adamascienza.com/summer-in-science/}{\textcolor{blue}{\textit{summer school}}}} for high school students I supervised and coordinated students alongside my daily schedule and am well prepared to supervise undergraduates or student hourly employees. \par%\vspace*{1mm}

Given my skill set, I believe I could make valuable contributions to the Avery lab and it would be an honor to contribute to the health and wellbeing of canine and feline patients. 
If you agree, I would appreciate the opportunity to discuss the position and my skills in more detail. 
Thank you for your time and consideration, I look forward to hearing from you soon.


 

\end{spacing}
%\vspace*{1mm} 
\makeletterclosing
\end{document}