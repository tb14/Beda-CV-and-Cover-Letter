%%%%%%%%%%%%%%%%%   HOW TO COMMENT!!! %%%%%%%%%%%%%%%%%%%%%%%
% Begin all comments particular to this document with `Thomas:' (unless it appears in `THOMAS GENERAL NOTES SECTION')
%To answer comments  please INDENT (just hit tab once or twice) and start with `Rabea:' 
%

%%%%%%%%%%%%%%%%%%%%%%%%% 		CHECKLIST		%%%%%%%%%%%%%%%%%%%%%%%%%
%
% 		Did you:
%				1) Check for the correct date?
%				2) Check that the recipeient's name and address were correct?
%				3) Copy into Word for spelling errors?
%				4) Use a word counter online to check for overused language?


%%%%%%%%%%%%%%%%%%%%%%%%%%%%%%%%%%%%%%%%%%%%%%%%%%%%%%%%%%%%%%%%%%%%%%%%%%%%%%%%%%%%%%%%%%%%%%%%%%%%

%																		BRANCH/VERSION 1 

%%%%%%%%%%%%%%%%%%%%%%%%%%%%%%%%%%%%%%%%%%%%%%%%%%%%%%%%%%%%%%%%%%%%%%%%%%%%%%%%%%%%%%%%%%%%%%%%%%%%




%%%%%%%%%%%%%%%%%%%%%		 DOCUMENT CLASS, PACKAGES 		%%%%%%%%%%%%%%%%%%%%%%%%%

\documentclass[11pt,a4paper,sans]{moderncv}

\usepackage{float, caption, fancyhdr, setspace, verbatim, xcolor}
\usepackage[normalem]{ulem} % Added this for strikeouts
\usepackage[scale=0.8, top = 1.8cm, bottom = 3cm]{geometry}                  


%%%%%%%%%%%%%%%%%%%%%%%%%%% 		FOOTER, HEADER, MODERNCV INPUTS		 %%%%%%%%%%%%%%%%%%%%%%%%%%%
\fancypagestyle{titlepage}{
\fancyhead{}
\fancyfoot[CE, CO]{\textcolor{gray}{Via Ramarini 32, Monterotondo RM, 00015 \thinspace\thinspace $\bullet$ \thinspace\thinspace\mobilephonesymbol +39~(388)~059~1507 \thinspace\thinspace $\bullet$ \thinspace\thinspace \faSkype \thinspace\thinspace \href{skype:rabea.jesser}{rabea.jesser} \thinspace\thinspace \thinspace\thinspace \emailsymbol \href{mailto:rabeajesser@web.de}{rabeajesser@web.de} \thinspace\thinspace $\bullet$ \thinspace\thinspace \faLinkedin \thinspace\thinspace \href{https://www.linkedin.com/in/rabeajesser/}{https://www.linkedin.com/in/rabeajesser/}}}}
\name{\vspace{-5mm}Rabea Jesser}{\vspace{3mm}}
\moderncvstyle{casual}                             
\moderncvcolor{blue}
\thispagestyle{titlepage}
% DONT FORGET TO UPDATE THIS!
\recipient{NREL National Bioenergy Center}{
%15013 Denver West Parkway, Golden, CO 80401
}
% DONT FORGET TO UPDATE THIS OR USE \today
\date{\today}
%%%%% IF YOU KNOW WHO YOU'RE SENDING THIS TO THEN GIVE THEIR NAME
\opening{Dear Dr. Pienkos,}
\closing{Sincerely,}
%\enclosure[Note]{M.Sc. Thesis, B.Sc., references and a more complete CV can also be sent upon request.}   


%%%%%%%%%%%%%%%%%%%% NOTES FOR CONTEXT OF LETTER, INFO ON ADDRESSEE %%%%%%%%%%%%%%%%%%%%%%%%%%%%%

% HEY! This is what I found under Philip Pleikos on this page: https://www.nrel.gov/bioenergy/algal-biofuels.html =>

%Establishing Testbeds
%
%As a member of the Algae Testbed Public-Private Partnership led by Arizona State University, we are focusing on establishing a sustainable network of regional testbeds to gather and share information within the algal research and development community, facilitate innovation, and accelerate growth of the algal biofuels and bioproducts industry. Our goals are to increase stakeholder access to high-quality, outdoor cultivation and laboratory facilities; support DOE's techno-economic, sustainability, and resource modeling activities; close critical knowledge gaps; and inform analyses of the state of technology for producing algal biofuels and bioproducts.
%
%Contact: Philip Pienkos
%
%There is also this:
%Algal Biomass Conversion
%
%NREL is developing novel process options to reduce the cost of algal biofuel production through more complete utilization of algal biomass. We have developed a low-cost, low-energy method to deconstruct algal biomass to allow for recovery and upgrading of lipids, carbohydrates, and proteins to biofuels and bioproducts.
%
%Contact: Philip Pienkos
%
% It sounds like if you want to really reach him, you should tailor some of the message around this: ``establishing a sustainable network of regional test beds'' to facilitate others' research and accelerate the industry



%%%%%%%%%%%%%%%% 			THOMAS GENERAL NOTES 			%%%%%%%%%%%%%%%%%%%%%%%%%%

% RABEA: Much better! Really, you did a great job of mentioning the specifics of your work! I am very impressed (not bullshitting you)
		%Rabea: Wow something good for a change :P, just kidding you said good stuff before already :)

%make sense `CHRONOLOGICALLY'. 


% Do you think it would be safe to mention ``Microbial Development and Metabolic Engineering''?
			%Rabea: It's also not that they brandmarked that term eh?

% ALWAYS ALWAYS ALWAYS keep this in mind when writing this: ``NREL advances the science and engineering of energy efficiency, sustainable transportation, and renewable power technologies and provides the knowledge to integrate and optimize energy systems.'' This is the language you should try to adopt when possible and especially when talking about why you want to work for them


%%%%%%%%%%%%%%%%%%%%%		BEGIN DOCUMENT		%%%%%%%%%%%%%%%%%%%%%%%%%%%%%%%%

\begin{document}     
\makelettertitle
\begin{spacing}{1.15}
%%%%%%%OUTLINE (I will comment this out later)%%%%%%%%%%%
%...............................................................................\\
%		1) Intro, we've been over how to do these\\
%		2) Hard science skills specific to EMBL + thesis (your work with algae and cyanobacteria in particular)\\
%		3) Soft science skills (maintaining a lab, presentation, etc.) + iGEM hard skills\\
%		4) Closing, you can do these well\\
%\par
%...............................................................................\\
        

% This is just still here because I want to have the phrases at hand: hands-on laboratory experience, strong work ethic and analytical thinking at the National Bioenergy Center to find affordable, practical alternatives to fossil fuels
With my strong background in microbiology, specifically with photosynthetic organisms, and a passion for renewable energy, I believe I could make a strong contribution to your lab and to algal research. 
My hands-on laboratory experience from my B.Sc., M.Sc. and current position with Europe's flagship laboratory for life sciences (EMBL) combined with a strong work ethic and analytical thinking would allow me to facilitate NBC's goal of accelerating the deployment and adoption of biofuels.
Likewise, my specific experience in microbial development and metabolic engineering in algae, cyanobacteria and \textit{E. coli} combined with the interpersonal skills I developed during the iGEM (international Genetically Engineered Machine) competition would also make me an excellent fit for your lab and ensure I could begin making contributions from the first day. 
%			Rabea: Well, this is ok but boring, there is no catching story just listing of what happened and some strong sounding adjectives, I don't know, we have to capture them with the first sentence no? So that they actually want to keep reading


\par\vspace*{1mm}

My journey with photosynthetic organisms started during my B.Sc. thesis when I got my hands on the diatom \textit{P. tricornutum} in which I expressed a recombinant antibody to characterize intracellular protein transport. 
% Unfortunately this data got lost in the depths of academics and never got published, a fact that I learned to improve in my M.Sc. {\href{https://www.ncbi.nlm.nih.gov/pubmed/29517395}{\textcolor{blue}{{\textit{(Publication)}}}}}. % 
% 		Thomas: I would drop this whole `unfortunately' thing. Better to just mention the B.Sc. work, talk about how you took those hard skills and brought them to your M.Sc., point to the publication and move on. Perhaps like so?:
			%Rabea: but it told a story
Later, during my M.Sc. thesis I was able to apply the laboratory skills I aquired during my B.Sc. to enlarge my methological skill set and alter the Synechocystis sp. PCC6803 genome via homologous recombination and transform a plasmid via conjugation to explore whether we could use the endogenous CRISPR-Cas system in metabolic engineering {\href{https://www.ncbi.nlm.nih.gov/pubmed/29517395}{\textcolor{blue}{{\textit{(Publication)}}}}}. 
In the process I became familiar with extracting and analyzing RNA \textit{in vitro} and from Synechocystis cultures, documenting pigment content via photometric measurements and producing genetic constructs and recombinant proteins in \textit{E. coli}. 
While completing my M.Sc. I participated in an interdisciplinary team of 20 students to develop a low-cost, multiplexed and label-free diagnostic tool as part of the Univerisity of Freiburg's team in the iGEM competition.
Within eight months of wet lab we produced high-value {\href{https://www.ncbi.nlm.nih.gov/pubmed/29803867}{\textcolor{blue}{\textit{data}}}}, presenting our team's findings on our {\href{http://2015.igem.org/Team:Freiburg}{\textcolor{blue}{\textit{website}}}} and at iGEM's international conference at MIT, where we were awarded a Gold medal and nominated for Best Health and Medicine Project, Best Innovation in Medicine and Best Wiki.
			%Rabea: One learns that he has a strong work ethic? 
This experience has left \sout{taught me that I have} me with a strong work ethic and able to efficiently identify the essentials from scientific findings and to present these in a comprehensive way to my team and a wider audience.
Once I finished \sout{with} my M.Sc., I began working in EMBL's Genetic Engineering Facility, which \sout {later} evolved into the Genetic and Viral Engineering Facility as soon as I started, so I quickly adapted and established new protocols for the production and quantification of viral vector tools in mammalian cell culture. 			
				%Rabea: 'later' that sounds like it took a year or something, and now I feel like the storyline is gone
In the beginning our two person team only provided recombinant AAV, but as my skills progressed, the scope of my and the lab's portfolio has expanded to include recombinant Lentivirus and we're in the process of adding HSV.\par
				%Rabea: That's exactly how it feels like: ``what's Rabea doin'?'' ``Just all our fucking AAV but who cares, let's start making her make HSV bullshyat, ya heard?'' I mean, we don't offer HSV because my skills progress but because the research groups want to use them and I'm rather forced to progress or just do whatever is written in any stupid protocol.

During iGEM we also successfully ran our own lab independently, including ordering reagents, communicating with companies and funding agencies for financial or material aid, making and sterilizing media and keeping the lab and our data organized. 
In my position with EMBL I have expanded on these skills and taken the initiative to extend the facility's database and automize calculations for protocols using Excel and Filemaker, helping streamline the lab and more consistently produce and record valuable data.
		%Rabea: now this sentence evolved to be terribly long
Additionally, I am continually applying techniques I learned from a professional development course on project management for scientists across more than 12 projects to deliver high quality final products in a timely fashion to our partners.
\par

Although my \sout{work on my theses and with} time at EMBL granted me extensive experience in the lab and provided \sout{for a fascinating (I wouldn't call this thing fascinating) challenge}room for growth, the research focus is just not after my own heart.
Your lab offers the opportunity both to continue \sout{my natural development as a microbiologist}my journey with photosynthetic organisms and help bring a sustainable and affordable alternative to fossil fuels to fruition. 
			%Rabea: Is any of this really natural? Naturally I would want to go live in a tree house and live off my homegrown potatoes...
%Given my skill set I believe I could make valuable contributions at the National Bioenergy Center and it would be an honor to contribute to your research into algal biofuels. 
			%Rabea: feels out of place now, can't we just leave it out?
If you agree, I would appreciate the opportunity to discuss any potential position and my contributions in more detail. 
Thank you for your time and consideration, I look forward to hearing from you.
			%Rabea: I think I don't want this anymore, can I just start the farm and be done with writing bullshit cover letters that noone wants to read anyways?


 

\end{spacing}
\makeletterclosing
\end{document}