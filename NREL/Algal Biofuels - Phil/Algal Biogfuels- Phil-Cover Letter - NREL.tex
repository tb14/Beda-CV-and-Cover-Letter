%%%%%%%%%%%%%%%%%   HOW TO COMMENT!!! %%%%%%%%%%%%%%%%%%%%%%%
% Begin all comments particular to this document with `Thomas:' (unless it appears in `THOMAS GENERAL NOTES SECTION')
%To answer comments  please INDENT (just hit tab once or twice) and start with `Rabea:' 
%

%%%%%%%%%%%%%%%%%%%%%%%%% 		CHECKLIST		%%%%%%%%%%%%%%%%%%%%%%%%%
%
% 		Did you:
%				1) Check for the correct date?
%				2) Check that the recipeient's name and address were correct?
%				3) Copy into Word for spelling errors?
%				4) Use a word counter online to check for overused language?


%%%%%%%%%%%%%%%%%%%%%%%%%%%%%%%%%%%%%%%%%%%%%%%%%%%%%%%%%%%%%%%%%%%%%%%%%%%%%%%%%%%%%%%%%%%%%%%%%%%%

%																		BRANCH/VERSION 1 

%%%%%%%%%%%%%%%%%%%%%%%%%%%%%%%%%%%%%%%%%%%%%%%%%%%%%%%%%%%%%%%%%%%%%%%%%%%%%%%%%%%%%%%%%%%%%%%%%%%%




%%%%%%%%%%%%%%%%%%%%%		 DOCUMENT CLASS, PACKAGES 		%%%%%%%%%%%%%%%%%%%%%%%%%

\documentclass[11pt,a4paper,sans]{moderncv}

\usepackage{float, caption, fancyhdr, setspace, verbatim, xcolor}
\usepackage[normalem]{ulem} % Added this for strikeouts
\usepackage[scale=0.8, top = 1.8cm, bottom = 3cm]{geometry}                  


%%%%%%%%%%%%%%%%%%%%%%%%%%% 		FOOTER, HEADER, MODERNCV INPUTS		 %%%%%%%%%%%%%%%%%%%%%%%%%%%
\fancypagestyle{titlepage}{
\fancyhead{}
\fancyfoot[CE, CO]{\textcolor{gray}{Via Ramarini 32, Monterotondo RM, 00015 \thinspace\thinspace $\bullet$ \thinspace\thinspace\mobilephonesymbol +39~(388)~059~1507 \thinspace\thinspace $\bullet$ \thinspace\thinspace \faSkype \thinspace\thinspace \href{skype:rabea.jesser}{rabea.jesser} \thinspace\thinspace \thinspace\thinspace \emailsymbol \href{mailto:rabeajesser@web.de}{rabeajesser@web.de} \thinspace\thinspace $\bullet$ \thinspace\thinspace \faLinkedin \thinspace\thinspace \href{https://www.linkedin.com/in/rabeajesser/}{https://www.linkedin.com/in/rabeajesser/}}}}
\name{\vspace{-5mm}Rabea Jesser}{\vspace{3mm}}
\moderncvstyle{casual}                             
\moderncvcolor{blue}
\thispagestyle{titlepage}
% DONT FORGET TO UPDATE THIS!
\recipient{NREL National Bioenergy Center}{15013 Denver West Parkway, Golden, CO 80401}
% DONT FORGET TO UPDATE THIS OR USE \today
\date{\today}
%%%%% IF YOU KNOW WHO YOU'RE SENDING THIS TO THEN GIVE THEIR NAME
\opening{Dear Dr. Phil Pienkos,}
\closing{Sincerely,}
%\enclosure[Note]{M.Sc. Thesis, B.Sc., references and a more complete CV can also be sent upon request.}   


%%%%%%%%%%%%%%%%%%%% NOTES FOR CONTEXT OF LETTER, INFO ON ADDRESSEE %%%%%%%%%%%%%%%%%%%%%%%%%%%%%

% HEY! This is what I found under Philip Pleikos on this page: https://www.nrel.gov/bioenergy/algal-biofuels.html =>

%Establishing Testbeds
%
%As a member of the Algae Testbed Public-Private Partnership led by Arizona State University, we are focusing on establishing a sustainable network of regional testbeds to gather and share information within the algal research and development community, facilitate innovation, and accelerate growth of the algal biofuels and bioproducts industry. Our goals are to increase stakeholder access to high-quality, outdoor cultivation and laboratory facilities; support DOE's techno-economic, sustainability, and resource modeling activities; close critical knowledge gaps; and inform analyses of the state of technology for producing algal biofuels and bioproducts.
%
%Contact: Philip Pienkos
%
%There is also this:
%Algal Biomass Conversion
%
%NREL is developing novel process options to reduce the cost of algal biofuel production through more complete utilization of algal biomass. We have developed a low-cost, low-energy method to deconstruct algal biomass to allow for recovery and upgrading of lipids, carbohydrates, and proteins to biofuels and bioproducts.
%
%Contact: Philip Pienkos
%
% It sounds like if you want to really reach him, you should tailor some of the message around this: ``establishing a sustainable network of regional test beds'' to facilitate others' research and accelerate the industry



%%%%%%%%%%%%%%%% 			THOMAS GENERAL NOTES 			%%%%%%%%%%%%%%%%%%%%%%%%%%

% Have you read anything they've published? Anything interesting? Anything you would want to work on? That may be a good segue. Any techniques they use a lot in which you are skilled.

% Ok, slowly but surely a general `outline' is starting to take shape in my mind (this is your cover letter so it's ultimately up to you, but . . .): 


% As I can't help with the #2 I'll focus on the #3 and #4

% It might also make more sense to squeeze all the science stuff into the 2nd paragraph and all the soft skills into the third paragraph

% In general I would say the main issue you could focus on writing-wise is flow. They're seeing this all for the first time, so it's important that the issues proceed logically from one another and they will be impressed if it is not just a bunch of different ideas bunched together (this is very hard to do and I am aware of that)

% WARNING- You should be aware, these people work both work for the U.S. government and work on this stuff everyday, this means: 
%		1) They cannot criticize the government and probably are loathe to criticize anyone. That means you can't insult fossil fuel companies and you can't insult governments.			%Rabea: What kind of democracy is that?
%		2) They aren't nearly as impressed by your committment to climate change (not on paper at least) as other places might be. For them, having worked on theories of climate change for so long, climate change is still a very `textbook' kind of problem to 					solve, not some major catastrophe. I mean, they think it's bad too, it's just most people don't work on these topics day-in-and-day-out so they are not fatigued by it all the time. These people are ALWAYS working on climate change.			%Rabea: So it seems like using that as a catch for the first paragraph is bullshit anyways so maybe I'd better use something else

%%%%%%%%%%%%%%%%%%%%%		BEGIN DOCUMENT		%%%%%%%%%%%%%%%%%%%%%%%%%%%%%%%%

\begin{document}     
\makelettertitle
\begin{spacing}{1.15}
%%%%%%%OUTLINE (I will comment this out later)%%%%%%%%%%%
%...............................................................................\\
%		1) Intro, we've been over how to do these\\
%		2) Hard science skills specific to EMBL + thesis (your work with algae and cyanobacteria in particular)\\
%		3) Soft science skills (maintaining a lab, presentation, etc.) + iGEM hard skills\\
%		4) Closing, you can do these well\\
%\par
%...............................................................................\\
        
%Although the overwhelming nature of climate change lends itself to despondency
%		Rabea: I would have never used these words, how about something like this:
This will not be the best piece of literature you will ever read---but I can take good care of your microorganisms and provide high-quality and reliable experimental data at the end of the day. And that is what counts, right? Sharing your passion for renewables, I will whole-heartedly invest my hands-on laboratory experience, strong work ethic and analytical thinking to find affordable, practical alternatives to fossil energy in order to build a sustainable future.
%				Which of the four groups here would you best fit in: https://www.nrel.gov/bioenergy/research.html  ?
%				Rabea: If I'd have to choose one it'd be under Biochemical Processes - Microbial Development and Metabolic Engineering, because it's interesting and I have the most experience in it
%		Rabea: Ok, how about this:
Combining my experience in microbial development and metabolic engineering in algae, cyanobacteria and \textit{E. coli} with the team spirit I aquired during the iGEM (international Genetically Engineered Machine) competition and the adaptive capacity I showed at Europe's flagship laboratory for the life sciences make me a perfect fit for NREL's Bioenergy Center.
Given the cutting edge research that takes place at your institute, it would be an honor to participate in your mission to transform energy through science.% 
%		Rabea: On their website they say: Mission and Programs		
%NREL advances the science and engineering of energy efficiency, sustainable transportation, and renewable power technologies and provides the knowledge to integrate and optimize energy systems. 
%		Then I also found a leaflet with the slogan: TRANSFORMING ENERGY---THROUGH SCIENCE 
\par\vspace*{1mm} %

My journey with photosynthetic organisms started during my B.Sc. thesis when I got my hands on the diatom \textit{P. tricornutum} in which I expressed a recombinant antibody to characterize intracellular protein transport. Unfortunately this data got lost in the depths of academics and never got published, a fact that I learned to improve in my M.Sc. {\href{https://www.ncbi.nlm.nih.gov/pubmed/29517395}{\textcolor{blue}{{\textit{(Publication)}}}}}. For this project I altered the Synechocystis sp. PCC6803 genome via homologous recombination and transformed a plasmid via conjugation to explore whether we can use the endogenous CRISPR-Cas system in metabolic engineering. In the process I became familiar with extracting and analyzing RNA \textit{in vitro} and from Synechocystis cultures, documenting pigment content via photometric measurements and producing genetic constructs and recombinant proteins in \textit{E. coli}. Again, unfortunately I had to move on before we could fully optimize the whole process.
%		Thomas: also, this guy works with algae and cyanobacteria, YOU HAVE WORKED WITH ALGAE AND CYANOBACTERIA. I would spend a lot more time talking about how your thesis prepares you to work in his lab and less time talking about BLS2							Did you learn any useful techniques? Spend any time outside your thesis with cyanobacteria? Here's where you can talk about it!				Rabea: Any better?
Fortunately my studies prepared me well for what was about to come. 
As I started my new postition at the Genetic Engineering Facility, it evolved into a  Genetic and Viral Engineering Facility and I established new standard protocols for the production and quantification of viral vector tools in mammalian cell culture. So we provide recombinant AAV and Lentivirus but are in the process of adding recombinant HSV to the facility's portfolio.
%		Rabea: Should I mention that we are only two people?
While completing my M.Sc. I participated in an interdisciplinary team of 20 students to develop a low-cost, multiplexed and label-free diagnostic tool as part of the Univerisity of Freiburg's team in the iGEM competition
\sout{submission}.
After only eight months of wet lab we produced high-value  {\href{https://www.ncbi.nlm.nih.gov/pubmed/29803867}{\textcolor{blue}{\textit{data}}}}, finally presenting our team's findings on our {\href{http://2015.igem.org/Team:Freiburg}{\textcolor{blue}{\textit{website}}}} and at an international conference, where we won a Gold medal and were nominated for Best Health and Medicine Project, Best Innovation in
Medicine and Best Wiki.
\par%\vspace*{1mm}
This experience has not only left me with a strong work ethic, but also with the ability to efficiently identify the essentials from scientific findings and to present these in a comprehensive way to my team and a wider audience.
During this time we \colorbox{green}{also}%		
%		Rabea: What does the green mean?
successfully ran our own lab independently, including ordering reagents, communicating with companies and funding agencies for financial or material aid, making and sterilizing media and keeping the lab and our data organized. 
In my current position I expanded on these skills and also took the initiative to improve the facility's database (FileMaker)---leaving me with excellent organizational skills I will invest to ensure accurate data recording and storage at the NBC.
In addition, I have attended a professional course on presenting with impact and project management for scientists and am currently applying these techniques to balance more than 12  projects at once while managing my time to deliver final products as quickly as possible.
\par%

Given my skill set I believe I could make valuable contributions to the National Bioenergy Center's team and it would be an honor to contribute to your research into algal biofuels. 
If you agree, I would appreciate the opportunity to discuss any potential position and my contributions in more detail. 
Thank you for your time and consideration, I look forward to hearing from you.


 

\end{spacing}
%\vspace*{1mm} 
\makeletterclosing
\end{document}