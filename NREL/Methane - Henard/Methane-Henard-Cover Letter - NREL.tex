%%%%%%%%%%%%%%%%%   RABEA READ ME!!! %%%%%%%%%%%%%%%%%%%%%%
%
% I am changing a few things just to clean it up. 
% Since we will mostly be working in the latex files from now on, most of our comments should be here too. 
% I will be moving towards only having the latex files in the github repository.
% In order to differentiate between old comments, and between your comments and mine, I will begin all my comments particular to this document with `Thomas:' (unless it appears in `THOMAS GENERAL NOTES SECTION')
% If you answer one of my comments with your own, please INDENT (just hit tab once or twice) and start with `Rabea:'  . . . This should make it easier to read these, thank you for your patience.
%
%
% ALSO: Please don't be angry, I made a lot of suggestions and not too many corrections. All in all I think your other cover letters were better at conveying why you would be a good fit. Perhaps we can talk over the phone about this soon? Love you!

%%%%%%%%%%%%%%%%%%%%%%%%% 		CHECKLIST		%%%%%%%%%%%%%%%%%%%%%%%%%
%
% 		Did you:
%				1) Check for the correct date?
%				2) Check that the recipeient's name and address were correct?
%				3) Copy into Word for spelling errors?
%				4) Use a word counter online to check for overused language?


%%%%%%%%%%%%%%%%%%%%%		 DOCUMENT CLASS, PACKAGES 		%%%%%%%%%%%%%%%%%%%%%%%%%

\documentclass[11pt,a4paper,sans]{moderncv}

\usepackage{float}
\usepackage{caption}
\usepackage{fancyhdr}
\usepackage{setspace}
\usepackage{verbatim}
\usepackage{xcolor} % Added this for highlights
\usepackage[normalem]{ulem} % Added this for strikeouts
\usepackage[scale=0.8, top = 1.8cm, bottom = 3cm]{geometry}                  

%%%% FIX THIS SO IT LOOKS NICER
% Thomas: Do you actually want to change the format of the footer or is the above comment outdated?			Rabea: Not sure what you mean here but I guess the answer is I don't want to change the format of the footer since I don't even know why you'd think I wanted to


%%%%%%%%%%%%%%%%%%%%%%%%%%% 		FOOTER, HEADER, MODERNCV INPUTS		 %%%%%%%%%%%%%%%%%%%%%%%%%%%

\fancypagestyle{titlepage}{
\fancyhead{}
\fancyfoot[CE, CO]{\textcolor{gray}{Via Ramarini 32, Monterotondo RM, 00015 \thinspace\thinspace $\bullet$ \thinspace\thinspace\mobilephonesymbol +39~(388)~059~1507 \thinspace\thinspace $\bullet$ \thinspace\thinspace \faSkype \thinspace\thinspace \href{skype:rabea.jesser}{rabea.jesser} \thinspace\thinspace \thinspace\thinspace \emailsymbol \href{mailto:rabeajesser@web.de}{rabeajesser@web.de} \thinspace\thinspace $\bullet$ \thinspace\thinspace \faLinkedin \thinspace\thinspace \href{https://www.linkedin.com/in/rabeajesser/}{https://www.linkedin.com/in/rabeajesser/}}}}
\name{\vspace{-5mm}Rabea Jesser}{\vspace{3mm}}
\moderncvstyle{casual}                             
\moderncvcolor{blue}
\thispagestyle{titlepage}
% DONT FORGET TO UPDATE THIS!
\recipient{National Bioenergy Center}{National Renewable Energy Laboratory\\15013 Denver West Parkway, Golden, CO 80401}
% DONT FORGET TO UPDATE THIS OR USE \today
\date{\today}
%%%%% IF YOU KNOW WHO YOU'RE SENDING THIS TO THEN GIVE THEIR NAME
\opening{Dear Dr. Calvin Henard,}
\closing{Sincerely,}
\enclosure[Note]{M.Sc. Thesis, B.Sc., references and a more complete CV can also be sent upon request.}   


%%%%%%%%%%%%%%%%%%%% NOTES FOR CONTEXT OF LETTER, INFO ON ADDRESSEE %%%%%%%%%%%%%%%%%%%%%%%%%%%%%

\begin{document}
%\large
%\centerline{\textbf{\underline{Adapting Methane with Yeast - Calvin Henard}}}
%\vspace{2.5mm}
%\normalsize
%{\doublespacing
%Rabea: These were just a collection of information for me in the same document for easy access so I make sure I can constantly look at what they are doing, I wasn't planning on having them in the acutual document in the end, I didn't comment them out because I do all my stuff in overleaf and look at the pdf at the same time as at the latex
%
%\begin{enumerate}
%\item Methylomicrobium alcaliphilum 20ZR as platform for production of fuels and value-added chemicals
%\item Model biochemical pathways
%\item Uses expertise in molecular biology and microbial genetics to develop algal, yeast, and bacterial biocatalyst for conversion of renewable substrates to biofuels and bioproducts
%\item Methane biocatalysis to fuels and chemicals
%\begin{enumerate}
%\item Methanogenesis and anaerobic digestion of lignocellulosic biomass and waste streams 
%\item Advanced genetic tool development for industrial microbes
%\item Isolation of environmental microbes with industrially promising characteristics
%\item The role of nitric oxide (NO) signaling in algal lipid accumulation

%\end{enumerate}
%\end{enumerate}
%}
%\vspace{5mm}\centerline{\noindent\rule{8cm}{0.4pt}}\vspace{5mm}


%%%%%%%%%%%%%%%% 			THOMAS GENERAL NOTES 			%%%%%%%%%%%%%%%%%%%%%%%%%%

% As you can tell I changed the outline of this document to make it easier to see comments/suggestions without the aid of a pdf
% I found the best way to insert comments within the paragraphs was to add a `%' after the end of the sentence(s) you are referring to and then use another `%' for the actual comment
% You can also probably tell I have yet to make any serious changes to the actual text . . . I will when I am not so tired/about to go to bed . . .  Please forgive me!!!!
% for added text, just so you can see it, I will highlight it in green, you can highlight yours in yellow (https://tex.stackexchange.com/questions/141569/highlight-textcolor-and-boldface-simultaneously#319000)
% for removed text I will strike it out (https://tex.stackexchange.com/questions/23711/strikethrough-text#23712)
% Whether you reject or accept my edits, get rid of the formatting. I only want to see your edits, not my left-over ones. So when I get this back, at least w.r.t. to the text I should only see your highlights and strikeouts, not mine.		Rabea: can you stop using acronyms? I don't understand and don't like them



%%%%%%%%%%%%%%%%%%%%%		BEGIN DOCUMENT		%%%%%%%%%%%%%%%%%%%%%%%%%%%%%%%%

       
\makelettertitle
\begin{spacing}{1.15}

I am fed up with being a contributor in the destruction of our livelihood the environment---I want to be part of the solution and have an abiding desire to contribute your research on biocatalysts to produce sustainable fuels or polymers.
With the strong background in microbiology I aquired during my B.Sc. and M.Sc. theses, the strong team spirit and work ethic I developed at the iGEM (international Genetically Engineered Machine) competition and the adaptive capacity I showed as a Technical Officer for Europe's flagship laboratory for the life sciences in Rome, I bring a solid background knowledge and the enthusiasm and eagerness to learn to make up for any deficiency. 
\par\vspace*{1mm} %

For my theses {\href{https://www.ncbi.nlm.nih.gov/pubmed/29517395}{\textcolor{blue}{{\textit{(Publication)}}}}} I manipulated and analyzed algae (\textit{P. tricornutum}, \textit{N. ocecanica}) and cyanobacteria (Synechocystis sp. PCC6803). 
More specifically, I used different transformation methods, analyzed their protein and RNA expression and explored the endogenous Synechocystis CRISPR-Cas system for appplication in metabolic engineering. 
The wide array of lectures and courses I took in biochemistry, translational biology and microbiology has  left me with a solid foundation to learn and develop new procedures and protocols in the lab. 
As an example, during a reorganization of my current lab, I consulted relevant literature to establish standard protocols for the production and quantification of viral vector tools in mammalian cell culture. 
Due to the reshaping I also started working in a BSL2 environment and am aware of the importance of stringently following safety guidelines and maintaining a safe working environment. 
Our facility currently only consists of my supervisor and me, meaning I not only produce high-quality viral vectors and genetic constructs independently, but also maintain a functional laboratory and took the initiative to improve the facility's database (FileMaker)---leaving me with excellent organizational skills I will invest to ensure accurate data recording and storage at the NBC.  
I have also attended a professional course on presenting with impact and project management for scientists and am able to apply these techniques to balance multiple projects---currently more than 12 at once---while managing my time to deliver final products as quickly as possible.

Working in an interdisciplinary team of 20 students during iGEM has prepared me to collaborate efficiently with NREL's scientists from different fields and to flourish in team environments while accomplishing common goals.
Within a month we agreed on an influential topic, and after only eight months of wet lab we produced high-value  {\href{https://www.ncbi.nlm.nih.gov/pubmed/29803867}{\textcolor{blue}{\textit{data}}}}, finally presenting our team's findings on our {\href{http://2015.igem.org/Team:Freiburg}{\textcolor{blue}{\textit{website}}}} and at an international conference. 
This experience has not only left me with a strong work ethic, but also with the ability to efficiently identify the essentials from scientific findings and to present these in a comprehensive way within the team, across departments and to a wider public.
During this time we successfully ran our own lab independently, including ordering reagents, communicating with companies and funding agencies for financial or material aid, making and sterilizing media and keeping the lab and our data organized. \par

Given my skillset I believe I could make valuable contributions to NBC's team and it would be an honor to be part of taking the research on methanotrophs to the lab. 
If you agree, I would appreciate the opportunity to discuss the position and my potential contributions in more detail.
Thank you for your time and consideration, I look forward to hearing from you. 

\end{spacing}
%\vspace*{1mm} 
\makeletterclosing
\end{document}