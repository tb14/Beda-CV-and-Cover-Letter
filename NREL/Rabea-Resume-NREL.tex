%%%%%%%%%%%%%%%%%%%%%%%%% 		CHECKLIST		%%%%%%%%%%%%%%%%%%%%%%%%%
%
% 		Did you:
%				1) Check for the correct date?
%				2) Check that the recipeient's name and address were correct?
%				3) Copy into Word for spelling errors?
%				4) Use a word counter online to check for overused language?


%%%%%%%%%%%%%%%%%%%%%		 DOCUMENT CLASS, PACKAGES, COMMANDS, SOURCING 		%%%%%%%%%%%%%%%%%%%%%%%%%

\documentclass[12pt,a4paper,sans]{moderncv}
\usepackage{graphicx, fancyhdr, setspace, verbatim, amsmath, acronym, times, epstopdf, amssymb, mathrsfs, latexsym, wrapfig, url, float, booktabs, microtype, multirow, textcomp, listings, blindtext}
\usepackage[scale=0.825, top = 1.75cm, bottom = 2.6cm]{geometry}
\usepackage[font={footnotesize,it}]{caption}  
\usepackage[english]{babel}\addto\captionsenglish  
{\renewcommand{\bibname}{References}}  
\usepackage[backend=bibtex,style=numeric, sorting = ydnt]{biblatex}  %backend=biber is 'better'  
\usepackage{csquotes} %Package biblatex Warning: 'babel/polyglossia' detected but 'csquotes' missing.  
\usepackage[T1]{fontenc}  
\usepackage{multicol}


\AfterPreamble{\usepackage{cleveref}} %\afterpreamble makes sure packages is only loaded after it does all the document class stuff, there is some sort of package conflict that makes this necessary

\addbibresource{publications-Rabea.bib}  

\defbibenvironment{bibliography}
  {\list
     {\printtext[labelnumberwidth]{% label format from numeric.bbx
        \printfield{labelprefix}%
        \printfield{labelnumber}}}
     {\setlength{\topsep}{0pt}% layout parameters from moderncvstyleclassic.sty
      \setlength{\labelwidth}{\hintscolumnwidth}%
      \setlength{\labelsep}{\separatorcolumnwidth}%
      \leftmargin\labelwidth%
      \advance\leftmargin\labelsep}%
      \sloppy\clubpenalty4000\widowpenalty4000}
  {\endlist}
  {\item}


\definecolor{dark-gray}{gray}{0.20}

% Try to create `triple' list
\newlength\Tripleitemmaincolumnwidth
\newlength\tripleitemmaincolumnwidth

\AtBeginDocument{%
\setlength\tripleitemmaincolumnwidth{% 
  \maincolumnwidth-2\hintscolumnwidth-2\separatorcolumnwidth}%
\setlength\tripleitemmaincolumnwidth{.333333\tripleitemmaincolumnwidth}%   
  }

\newcommand*{\cvtripleitem}[7][.25em]{%
 \cvitem[#1]{\textbf{#2}}{%
   \begin{minipage}[t]{\tripleitemmaincolumnwidth}\emph{#3}\end{minipage}%
   \hfill
   \begin{minipage}[t]{\hintscolumnwidth}\raggedleft\hintstyle{\textbf{#4}}\end{minipage}%
   \hspace*{\separatorcolumnwidth}%
   \begin{minipage}[t]{\tripleitemmaincolumnwidth}\emph{#5}\end{minipage}%
   \hfill
   \begin{minipage}[t]{\hintscolumnwidth}\raggedleft\hintstyle{\textbf{#6}}\end{minipage}%
   \hspace*{\separatorcolumnwidth}%
   \begin{minipage}[t]{\tripleitemmaincolumnwidth}\emph{#7}\end{minipage}%
   }%
}

%%%%%%%%%%%%%%%%%%%%%%%%%%% 		FOOTER, HEADER, MODERNCV INPUTS		 %%%%%%%%%%%%%%%%%%%%%%%%%%%

\moderncvstyle{casual}                             
\moderncvcolor{blue}                              
\nopagenumbers{}                          

\def\changemargin#1#2{\list{}{\rightmargin#2\leftmargin#1}\item[]}
\let\endchangemargin=\endlist 


\pagestyle{fancy}
\fancyhead{}
\fancyfoot{}
\fancypagestyle{titlepage}{
\fancyhead{}
\fancyfoot[CE, CO]{\textcolor{gray}{Via Ramarini 32, Monterotondo RM, 00015 \thinspace\thinspace $\bullet$ \thinspace\thinspace\mobilephonesymbol +39~(388)~059~1507 \thinspace\thinspace $\bullet$ \thinspace\thinspace \faSkype \thinspace\thinspace \href{skype:rabea.jesser}{rabea.jesser} \thinspace\thinspace \thinspace\thinspace \emailsymbol \href{mailto:rabeajesser@web.de}{rabeajesser@web.de} \thinspace\thinspace $\bullet$ \thinspace\thinspace \faLinkedin \thinspace\thinspace \href{https://www.linkedin.com/in/rabeajesser/}{https://www.linkedin.com/in/rabeajesser/}}}}

\name{Rabea Jesser ---}{ CV}


%%%%%%%%%%%%%%%% 			THOMAS GENERAL NOTES 			%%%%%%%%%%%%%%%%%%%%%%%%%%
% Rabea: I cut it down to make it a reasonable length, what do you think?
% 		Thomas: I like it, just see my comments there (a lot are left over), I think we can still make it better but it's come a long way
%		Thomas: I don't think you need to worry too much about length, the guys at NREL did not seem to care about that
%		Thomas: other than that I would say we just nee to make this paragraph flow a bit more smoothly (although as a summary paragraph it will to a certian extent always remain disjointed)

%%%%%%%%%%%%%%%%			RABEA GENERAL NOTES			
%%% 	
%%%%%%%%%%%%%%%%%%%%%%%%


%%%%%%%%%%%%%%%%%%%%%		BEGIN DOCUMENT		%%%%%%%%%%%%%%%%%%%%%%%%%%%%%%%%

\begin{document}
%
\thispagestyle{titlepage}
%
\renewcommand*{\namefont}{\fontsize{18}{2}\mdseries\upshape}
\renewcommand*{\sectionfont}{\fontsize{16}{8}\mdseries\upshape}
\renewcommand*{\subsectionfont}{\fontsize{14}{5}\mdseries\upshape}
%
\makecvtitle{}\vspace*{-13mm}
%
% SUMMARY PARAGRAPH
%
\begin{spacing}{1.2}
\color{dark-gray}{
Photosynthetic organisms will play a vital role in transitioning towards a sustainable economy and as a dedicated researcher with a strong practical and theoretical background, I am well equipped to help make this a reality. 
During my theses and time at Europe's flagship laboratory for the life sciences (EMBL) I have honed my skills in genetically engineering  \textit{E. coli} as well as algae and cyanobacteria.%
%  Thomas: Are there any techniques specific to these guys that you also see their lab using?  Techniques you are really good at?
Through these experiences and my time competing in iGEM, I have  developed an ability to function both independently and as part of an interdisciplinary team. 
% Thomas: an eye for details and the analytical skills necessary to consult relevant literature to establish new protocols?
All this combined with a strong work ethic and a passion for bioenergy make me an excellent candidate for the National Bioenergy Center.
% Thomas: and I believe working on your cutting edge research would bring me closer to my goals of making bioenergy a reliable and affordable alternative to fossil fuels.
}
%
% VOCATIONAL EXPERIENCE
%
%\vspace{2mm}
\section{Vocational Experience}
\cvitem{2017--present}{\emph{European Molecular Biology Laboratory (EMBL)}, Monterotondo, Italy}
\cvitem{Job Title}{\textbf{Technical Officer}, Genetic and Viral Engineering Facility}
\cvitem{Achievements}{
\begin{itemize}  
		\item Produce high-quality genetic constructs in \textit{E. coli }and viral vector tools like recombinant AAV or lentivirus in HEK cells
		\item Establish new protocols for the production of viral vector tools 
		\item Automate calculations for experimental procedures using FileMaker database and Microsoft Excel
		\item Improve FileMaker database and accurately manage sample data 
		\item Keep laboratory running smoothly (monitor lab supply, equipment and instruments, prepare stock solutions and media)
\end{itemize}}
\vspace{-6mm}
%
% RESEARCH EXPERIENCE
\vspace{-2mm}
\section{Research Experience}
\subsection{M.Sc. Thesis --- Genetics \& Experimental Bioinformatics}
\cvitem{Topic}{\emph{Biochemical analysis of the Cas6-1 RNA endonuclease associated with the subtype I-D CRISPR-Cas system in Synechocystis sp. PCC 6803.}{\:\href{https://www.ncbi.nlm.nih.gov/pubmed/29517395}{\textcolor{blue}{\textit{(Publication)}}}}}
\cvitem{Motivation}{Explore applicability of endogenous Synechocystis CRISPR/Cas system in metabolic engineering}
\cvitem{Achievements}{
\begin{itemize}  
		\item Explored the endogenous Synechocystis CRISPR-Cas system for appplication in metabolic engineering 
		\item Successfully generated Synechocystis mutant strains via homologous recombination and transformed them with plasmid DNA via 
		\item Analyzed crRNA processing \textit{in vitro} with mutant Cas6 proteins expressed in \textit{E. coli} and \textit{in vitro} transcribed RNA 
		\item Analyzed crRNA processing \textit{in vivo} with inducible artificial crRNA and inducible mutant Cas6 proteins
		\item Assembled all genetic constructs in \textit{E. coli} using Gibson cloning
\end{itemize}}
%\vspace{3mm}
\subsection{iGem Competition }
\cvitem{Topic}{\emph{Multiplexed antibody detection from blood sera by immobilization of in vitro expressed antigens and label-free readout via imaging reflectometric interferometry (iRIf).}{\:\href{https://www.ncbi.nlm.nih.gov/pubmed/29803867}{\textcolor{blue}{\textit{(Publication)}}}}{\:\href{http://2015.igem.org/Team:Freiburg}{\textcolor{blue}{\textit{(Team Website)}}}}}
\cvitem{Motivation}{Allow for a cheap and quick pre-test that screens for multiple diseases}
\cvitem{Achievements}{
\begin{itemize}  
		\item Developed a prototype for multiplexed, microfluidics-based, label-free diagnostic tool 
		\item Analyzed literature to find a commonly agreed on high-impact project
		\item Overexpressed and purified proteins \textit{E. coli} via NiNTA column
		\item Successfully organized ourself as an interdisciplinary team to ensure availability of sufficient funding and reagents and a functional lab
		\item Taught myself adobe illustrator and designed most of the explanatory illustrations for the  {\:\href{http://2015.igem.org/Team:Freiburg}{\textcolor{blue}{\textit{website}}}} 
		\item Accurately recorded and communicated results comprehensively for all team members, to a crowdfunding community and to a jury and fellow students at the final international conference
		\item Contributed to{\:\href{https://www.ncbi.nlm.nih.gov/pmc/articles/PMC4777433/pdf/pone.0150182.pdf}{\textcolor{blue}{\textit{interlab study}}}}
\end{itemize}
}
\vspace{-6mm}
\cvitem{Awards}{Gold medal; nominated for Best Health and Medicine Project, Best Innovation in Medicine and Best Wiki.}
%\vspace{2mm}
\subsection{B.Sc. Thesis --- Cell Biology}
\cvitem{Topic}{\emph{\textit{Nannochloropsis oceanica} as an expression system for recombinant proteins and studies on protein transport across the periplastidal membrane of \textit{Phaeodactylum tricornutum}.}}
%		%Rabea: I didn't really get anywhere on the N. oceanica part, should I still metion it?
%				Thomas: yes I would
\cvitem{Motivation}{Enlargen molecular toolbox for \textit{P. tricornutum} and explore \textit{N. oceanica} as expression platform for recombinant antibodies}
\cvitem{Achievements}{
\begin{itemize}  
		\item Expressed recombinant single-chain antibody in\textit{ P. tricornutum} to study intracellular protein transport
		\item Assembled genetic constructs in \textit{E. coli} using traditional cloning methods
\end{itemize}}
%
% EDUCATION
\cventry{2014--2016}{M.Sc. Biology}{\href{http://www.bio.uni-freiburg.de/studies/degree-programs/master-en/translational_biology}{\textcolor{blue}{Albert-Ludwigs-Universit\"at Freiburg}}}{Germany}{\textit{GPA 4.0}}{}
\cventry{2011--2014}{B.Sc. Biology}{\href{ https://www.uni-marburg.de/fb17/index_html-en?set_language=en?}{\textcolor{blue}{Philipps-Universit\"at Marburg}}}{Germany}{\textit{GPA 4.0}}{}
% LANGUAGES
%
\section{Languages}
\cvtripleitem{German:}{Native}{English:}{Fluent}{Italian:}{Fluent}
%
% COMPUTER SKILLS
%
\section{Computer Skills}
\cvitemwithcomment{Office}{Microsoft Excel, Word, PowerPoint}{very proficient}
\cvitemwithcomment{Biology}{Geneious, SnapGene}{very proficient}
\cvitemwithcomment{Other}{Adobe Illustrator, FileMaker database, \LaTeX, GitKraken}{basic}{}
%

% PUBLICATIONS 
%

% Thomas: RABEA! So you had commented this out (lack of space?) but they said you should ABSOLUTELY have your publications on here, and were much less concerned with a 2 page CV (one of them even said now that it's online he rarely even notices how long they are)
%		Rabea: No, it didn't work and I thought they are already in the links, do I need another file for it to work or connect it differently somehow?
%				Thomas: I will work on this, do you see an error or what?
\nocite{Article3,Article2,Article1}
\printbibliography[title={Publications}]
%
% REFERENCES
%
\section{References}
\cvitem{EMBL}{\textbf{Dr. James Sawitzke:} \emph{Head of Genetic and Viral Engineering Facility, EMBL Rome --- Italy} \emailsymbol \href{mailto:james.sawitzke@embl.it}{james.sawitzke@embl.it} \thinspace \phonesymbol +39 06 90091 268 \thinspace \homepagesymbol\textcolor{blue}{\url{https://www.embl.it/services/genetic-and-viral-engineering-facility/index.html}}}
\cvitem{M.Sc. Thesis}{\textbf{Dr. Wolfgang Hess:} \emph{Professor for Genetics \& Experimental Bioinformatics, University Freiburg  --- Germany} \emailsymbol \href{mailto:wolfgang.hess@biologie.uni-freiburg.de}{wolfgang.hess@biologie.uni-freiburg.de} \thinspace \phonesymbol +49 761 203-2796 \thinspace \homepagesymbol\textcolor{blue}{\url{http://www.cyanolab.de/}}}
\cvitem{iGem}{\textbf{Dr. Maximilian Ulbrich:} \emph{Group leader at Centre for Biological Signalling Studies (BIOSS), University Freiburg --- Germany} \emailsymbol \href{mailtp:max.ulbrich@bioss.uni-freiburg.de}{ max.ulbrich@bioss.uni-freiburg.de} \thinspace \phonesymbol +49 761 203 97183 \thinspace \homepagesymbol\textcolor{blue}{\url{http://www.ulbrich-lab.com/}}}
\cvitem{iGem}{\textbf{Dr. Nicole Gensch:} \emph{Laboratory manager of the Toolbox, BIOSS, University Freiburg --- Germany} \emailsymbol nicole.gensch@bioss.uni-freiburg.de \thinspace \phonesymbol +49 761 203 97225  \thinspace \homepagesymbol \textcolor{blue}{\url{http://www.bioss.uni-freiburg.de/de/toolbox/toolbox-home/}}}

\end{spacing}
\thispagestyle{titlepage}
\end{document}