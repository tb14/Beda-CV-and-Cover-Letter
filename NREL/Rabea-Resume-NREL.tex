%%%%%%%%%%%%%%%%%%%%%%%%% 		CHECKLIST		%%%%%%%%%%%%%%%%%%%%%%%%%
%
% 		Did you:
%				1) Check for the correct date?
%				2) Check that the recipeient's name and address were correct?
%				3) Copy into Word for spelling errors?
%				4) Use a word counter online to check for overused language?


%%%%%%%%%%%%%%%%%%%%%		 DOCUMENT CLASS, PACKAGES, COMMANDS, SOURCING 		%%%%%%%%%%%%%%%%%%%%%%%%%

\documentclass[12pt,a4paper,sans]{moderncv}
\usepackage{graphicx, fancyhdr, setspace, verbatim, amsmath, acronym, times, epstopdf, amssymb, mathrsfs, latexsym, wrapfig, url, float, booktabs, microtype, multirow, textcomp, listings, blindtext}
\usepackage[scale=0.825, top = 1.75cm, bottom = 2.6cm]{geometry}
\usepackage[font={footnotesize,it}]{caption}  
\usepackage[english]{babel}\addto\captionsenglish  
{\renewcommand{\bibname}{References}}  
\usepackage[backend=bibtex,style=numeric, sorting = ydnt]{biblatex}  %backend=biber is 'better'  
\usepackage{csquotes} %Package biblatex Warning: 'babel/polyglossia' detected but 'csquotes' missing.  
\usepackage[T1]{fontenc}  
\usepackage{multicol}


\AfterPreamble{\usepackage{cleveref}} %\afterpreamble makes sure packages is only loaded after it does all the document class stuff, there is some sort of package conflict that makes this necessary

\addbibresource{publications-Rabea.bib}  

\defbibenvironment{bibliography}
  {\list
     {\printtext[labelnumberwidth]{% label format from numeric.bbx
        \printfield{labelprefix}%
        \printfield{labelnumber}}}
     {\setlength{\topsep}{0pt}% layout parameters from moderncvstyleclassic.sty
      \setlength{\labelwidth}{\hintscolumnwidth}%
      \setlength{\labelsep}{\separatorcolumnwidth}%
      \leftmargin\labelwidth%
      \advance\leftmargin\labelsep}%
      \sloppy\clubpenalty4000\widowpenalty4000}
  {\endlist}
  {\item}


\definecolor{dark-gray}{gray}{0.20}

% Try to create `triple' list
\newlength\Tripleitemmaincolumnwidth
\newlength\tripleitemmaincolumnwidth

\AtBeginDocument{%
\setlength\tripleitemmaincolumnwidth{% 
  \maincolumnwidth-2\hintscolumnwidth-2\separatorcolumnwidth}%
\setlength\tripleitemmaincolumnwidth{.333333\tripleitemmaincolumnwidth}%   
  }

\newcommand*{\cvtripleitem}[7][.25em]{%
 \cvitem[#1]{\textbf{#2}}{%
   \begin{minipage}[t]{\tripleitemmaincolumnwidth}\emph{#3}\end{minipage}%
   \hfill
   \begin{minipage}[t]{\hintscolumnwidth}\raggedleft\hintstyle{\textbf{#4}}\end{minipage}%
   \hspace*{\separatorcolumnwidth}%
   \begin{minipage}[t]{\tripleitemmaincolumnwidth}\emph{#5}\end{minipage}%
   \hfill
   \begin{minipage}[t]{\hintscolumnwidth}\raggedleft\hintstyle{\textbf{#6}}\end{minipage}%
   \hspace*{\separatorcolumnwidth}%
   \begin{minipage}[t]{\tripleitemmaincolumnwidth}\emph{#7}\end{minipage}%
   }%
}

%%%%%%%%%%%%%%%%%%%%%%%%%%% 		FOOTER, HEADER, MODERNCV INPUTS		 %%%%%%%%%%%%%%%%%%%%%%%%%%%

\moderncvstyle{casual}                             
\moderncvcolor{blue}                              
\nopagenumbers{}                          

\def\changemargin#1#2{\list{}{\rightmargin#2\leftmargin#1}\item[]}
\let\endchangemargin=\endlist 


\pagestyle{fancy}
\fancyhead{}
\fancyfoot{}
%\fancyhead[R]{\textcolor{gray}{$\sim$ \textbf{Rabea Jesser} $\sim$}}
\fancypagestyle{titlepage}{
\fancyhead{}
\fancyfoot[CE, CO]{\textcolor{gray}{Via Ramarini 32, Monterotondo RM, 00015 \thinspace\thinspace $\bullet$ \thinspace\thinspace\mobilephonesymbol +39~(388)~059~1507 \thinspace\thinspace $\bullet$ \thinspace\thinspace \faSkype \thinspace\thinspace \href{skype:rabea.jesser}{rabea.jesser} \thinspace\thinspace \thinspace\thinspace \emailsymbol \href{mailto:rabeajesser@web.de}{rabeajesser@web.de} \thinspace\thinspace $\bullet$ \thinspace\thinspace \faLinkedin \thinspace\thinspace \href{https://www.linkedin.com/in/rabeajesser/}{https://www.linkedin.com/in/rabeajesser/}}}}

\name{Rabea Jesser ---}{ CV}


%%%%%%%%%%%%%%%% 			THOMAS GENERAL NOTES 			%%%%%%%%%%%%%%%%%%%%%%%%%%
%
% Thomas: I love the changes! The additional detail is great!
%
% Thomas: Also, it's easier for me to see when you reply to my comments on a new, indented line
%				Thomas: Like this!
%
% Thomas: Do you want your address stuff to be at the bottom of every page or just the first one?		Rabea: Maybe every page is better
%
%
% Thomas: I think the summary paragraph in the beginning is equivalent to a combo of the first and last paragraph in the cover letter: 1) This is why you want me + This is why I want you (in that order of importance). 
%			I changed yours, let me know what you think.		%I cut it down to make it a reasonable length, what do you think?
%
% Summary paragraph: 4-5 easily digestible (readable) sentences.
%1) I know what you're working on and here's roughly why I could help; 
%2) Here are some hard skills I have that you would find useful; 
%3) Here are some soft skills I could use anywhere that you would find useful; 
%4) I am really passionate about what you do and bring a great attitude and work ethic 
%5) You all are my dream job, please please please take me (hysterical sobbing)! 



%%%%%%%%%%%%%%%%			RABEA GENERAL NOTES			
%%% 	
%%%%%%%%%%%%%%%%%%%%%%%%


%%%%%%%%%%%%%%%%%%%%%		BEGIN DOCUMENT		%%%%%%%%%%%%%%%%%%%%%%%%%%%%%%%%

\begin{document}
%
\thispagestyle{titlepage}
%
\renewcommand*{\namefont}{\fontsize{18}{2}\mdseries\upshape}
\renewcommand*{\sectionfont}{\fontsize{16}{8}\mdseries\upshape}
\renewcommand*{\subsectionfont}{\fontsize{14}{5}\mdseries\upshape}
%
\makecvtitle{}\vspace*{-13mm}
%
% SUMMARY PARAGRAPH
%
\begin{spacing}{1.2}
\color{dark-gray}{
Photosynthetic organisms will play a vital role in transitioning towards a sustainable economy and as a dedicated researcher with a strong practical and theoretical background, I am well equipped to help make this a reality. 
%               maybe something like a specific technique (assays?)), and below you list all the times you did X and in the cover letter you write ``While working on my thesis I honed my ability in ``assays'', a skill I further improved while at EMBL and one which I                         %		     could employ successfully at NREL (or something along those lines)
During my theses and time at Europe's flagship laboratory for the life sciences (EMBL) I have honed my skills in genetically engineering  \textit{E. coli} as well as algae and cyanobacteria.%
%  work on inter-sentence flow
Through these experiences and my time competing in iGEM, I have  developed an ability to function both independently and as part of an interdisciplinary team. %, an eye for details and the analytical skills necessary to consult relevant literature to establish new protocols. %
% a few soft skills you think they might want to hear about. We should also be careful about talking about things you never mention again. This is just the intor for what you explain later below and in your cover 				letter
% Thomas: might also want to squeeze in your computer skills? Dunno.
All this combined with a strong work ethic and a passion for bioenergy make me an excellent candidate for the National Bioenergy Center.
%and I believe working on your cutting edge research would bring me closer to my goals of making bioenergy a reliable and affordable alternative to fossil fuels.
}
%
% VOCATIONAL EXPERIENCE
%
\vspace{2mm}
\section{Vocational Experience}
\cvitem{2017--present}{\emph{European Molecular Biology Laboratory (EMBL)}, Monterotondo, Italy}
\cvitem{Job Title}{\textbf{Technical Officer}, Genetic and Viral Engineering Facility}
\cvitem{Achievements}{
\begin{itemize}  
		\item Produce high-quality genetic constructs in \textit{E. coli }and viral vector tools like recombinant AAV or lentivirus in HEK cells
		\item Establish new protocols for the production of viral vector tools 
		\item Automate calculations for experimental procedures using FileMaker database and Microsoft Excel
		\item Improve FileMaker database and accurately manage sample data 
		\item Keep laboratory running smoothly (monitor lab supply, equipment and instruments, prepare stock solutions and media)
\end{itemize}}
\vspace{-6mm}
%
% RESEARCH EXPERIENCE
%
\section{Research Experience}
\subsection{M.Sc. Thesis --- Genetics \& Experimental Bioinformatics}
\cvitem{Topic}{\emph{Biochemical analysis of the Cas6-1 RNA endonuclease associated with the subtype I-D CRISPR-Cas system in Synechocystis sp. PCC 6803.}{\:\href{https://www.ncbi.nlm.nih.gov/pubmed/29517395}{\textcolor{blue}{\textit{(Publication)}}}}}
\cvitem{Achievements}{
\begin{itemize}  
		\item Explored the endogenous Synechocystis CRISPR-Cas system for appplication in metabolic engineering 
		\item Successfully generated Synechocystis mutant strains via homologous recombination and transformed them with plasmid DNA via 
		\item Analyzed crRNA processing \textit{in vitro} with mutant Cas6 proteins expressed in \textit{E. coli} and \textit{in vitro} transcribed RNA 
		\item Analyzed crRNA processing \textit{in vivo} with inducible artificial crRNA and inducible mutant Cas6 proteins
		\item Assembled all genetic constructs in \textit{E. coli} using Gibson cloning
\end{itemize}}
\vspace{3mm}
\subsection{iGem 2015 Competition }
\cvitem{Topic}{\emph{Multiplexed antibody detection from blood sera by immobilization of in vitro expressed antigens and label-free readout via imaging reflectometric interferometry (iRIf).}{\:\href{https://www.ncbi.nlm.nih.gov/pubmed/29803867}{\textcolor{blue}{\textit{(Publication)}}}}{\:\href{http://2015.igem.org/Team:Freiburg}{\textcolor{blue}{\textit{(Team Website)}}}}}
\cvitem{Achievements}{
% Thomas: What might also be important here: what's the goal? why were you all doing this? To allow for a cheap, reliable and quick method to test multiple diseases, right? That's pretty fucking cool! You should mention that!
\begin{itemize}  
		\item Developed a prototype for multiplexed, microfluidics-based, label-free diagnostic tool 
		\item Collected and analyzed literature to find a commonly agreed on high-impact project
		\item Over-expressed and purified proteins \textit{E. coli} % Overexpressed spelling? Using what technique?
		\item Successfully managed lab as a team % I would try and add more detail here: interdisciplinary, 20 students, were you the manager?
%				Thomas:	 These updates are so great! So specific! It's wonderful, really. The one above would be one of the few which is still a bit vague, can you add any more detail? If not I think it'll be ok.		
		\item Designed explanatory illustrations for the  {\:\href{http://2015.igem.org/Team:Freiburg}{\textcolor{blue}{\textit{website}.}}} % with what program(s)?
		\item Accurately recorded and communicated results comprehensively for all team members as well as to a broad audience% Thomas: Who is the `broader audience' here? Was it at the international iGEM conference?
		\item Contributed to{\:\href{https://www.ncbi.nlm.nih.gov/pmc/articles/PMC4777433/pdf/pone.0150182.pdf}{\textcolor{blue}{\textit{interlab study}.}}}
			%	Rabea: this is the interlab study where we only contrubited, I moved our real publication to after the Topic
%					Thomas: Ok, cool, as long as they can see the difference (or rather that they are two different works) it'll be fine				
\end{itemize}
%\begin{multicols}{2}
%\begin{itemize}
%    \item Successfully managed lab as a team
%    \item Contributed to{\:\href{https://www.ncbi.nlm.nih.gov/pmc/articles/PMC4777433/pdf/pone.0150182.pdf}{\textcolor{blue}{\textit{interlab study}.}}}
%\end{itemize}
%\end{multicols}
% Thomas: Tried to use this to save space (set up two columns for your shorter points
}
\vspace{-6mm}
\cvitem{Awards}{Gold medal; nominated for Best Health and Medicine Project, Best Innovation in Medicine and Best Wiki.}
\vspace{2mm}
\subsection{B.Sc. Thesis --- Cell Biology}
\cvitem{Topic}{\emph{\textit{Nannochloropsis oceanica} as an expression system for recombinant proteins and studies on protein transport across the periplastidal membrane of \textit{Phaeodactylum tricornutum}.}}
\cvitem{Achievements}{
\begin{itemize}  
		\item Expressed recombinant single-chain antibody in\textit{ P. tricornutum} to study intracellular protein transport
		\item Extracted and analyzed the size of a periplastidal membrane complex from \textit{P. tricornutum} 
		\item Assembled genetic constructs in \textit{E. coli} using traditional cloning methods
\end{itemize}}
%
% EDUCATION
\thispagestyle{titlepage}
%
\section{Education}
\cventry{2014--2016}{M.Sc. Biology}{\href{http://www.bio.uni-freiburg.de/studies/degree-programs/master-en}{\textcolor{blue}{Albert-Ludwigs-Universit\"at Freiburg}}}{Germany}{\textit{GPA 4.0}}{}
\cventry{2011--2014}{B.Sc. Biology}{\href{https://www.uni-marburg.de/fb17/fachgebiete/zellbio/zellbioi?language_sync=1}{\textcolor{blue}{Philipps-Universit\"at Marburg}}}{Germany}{\textit{GPA 4.0}}{}
%
% LANGUAGES
%
\section{Languages}
\cvtripleitem{German:}{Native}{English:}{Fluent}{Italian:}{Fluent}
%
% COMPUTER SKILLS
%
\section{Computer Skills}
\cvitemwithcomment{Office}{Microsoft Excel, Word, PowerPoint}{very proficient}
\cvitemwithcomment{Biology}{Geneious, SnapGene}{very proficient}
\cvitemwithcomment{Other}{Adobe Illustrator, FileMaker database, \LaTeX, GitKraken}{basic}{}
%
% Thomas: I would consider updating this and including it . . . You decide		%		Rabea: I cannot think of anything relevant for them from my volunteering
%\section{Volunteering}
%\cvitem{}{Community garden at EMBL, organized summer camp for 36 young women with 11 co-workers}
%
%
% PUBLICATIONS 
%

% Thomas: RABEA! So you had commented this out (lack of space?) but they said you should ABSOLUTELY have your publications on here, and were much less concerned with a 2 page CV (one of them even said now that it's online he rarely even notices how long they are)
\nocite{Article3,Article2,Article1}
\printbibliography[title={Publications}]
%
% REFERENCES
%
\section{References}
\cvitem{EMBL}{\textbf{Dr. James Sawitzke:} \emph{Head of Genetic and Viral Engineering Facility, EMBL Rome --- Italy} \emailsymbol \href{mailto:james.sawitzke@embl.it}{james.sawitzke@embl.it} \thinspace \phonesymbol +39 06 90091 268 \thinspace \homepagesymbol\textcolor{blue}{\url{https://www.embl.it/services/genetic-and-viral-engineering-facility/index.html}}}
\cvitem{M.Sc. Thesis}{\textbf{Dr. Wolfgang Hess:} \emph{Professor for Genetics \& Experimental Bioinformatics, University Freiburg  --- Germany} \emailsymbol \href{mailto:wolfgang.hess@biologie.uni-freiburg.de}{wolfgang.hess@biologie.uni-freiburg.de} \thinspace \phonesymbol +49 761 203-2796 \thinspace \homepagesymbol\textcolor{blue}{\url{http://www.cyanolab.de/}}}
\cvitem{iGem}{\textbf{Dr. Maximilian Ulbrich:} \emph{Group leader at Centre for Biological Signalling Studies (BIOSS), University Freiburg --- Germany} \emailsymbol \href{mailtp:max.ulbrich@bioss.uni-freiburg.de}{ max.ulbrich@bioss.uni-freiburg.de} \thinspace \phonesymbol +49 761 203 97183 \thinspace \homepagesymbol\textcolor{blue}{\url{http://www.ulbrich-lab.com/}}}
\cvitem{iGem}{\textbf{Dr. Nicole Gensch:} \emph{Laboratory manager of the Toolbox, BIOSS, University Freiburg --- Germany} \emailsymbol nicole.gensch@bioss.uni-freiburg.de \thinspace \phonesymbol +49 761 203 97225  \thinspace \homepagesymbol \textcolor{blue}{\url{http://www.bioss.uni-freiburg.de/de/toolbox/toolbox-home/}}}

\end{spacing}
\thispagestyle{titlepage}
\end{document}