\documentclass[11pt,a4paper,sans]{moderncv}

\usepackage{float}
\usepackage{caption}
\usepackage{fancyhdr}
\usepackage{setspace}
\usepackage{verbatim}
\usepackage[scale=0.8, top = 1.8cm, bottom = 3cm]{geometry}

\moderncvstyle{casual}                             
\moderncvcolor{blue}                               


%%%% FIX THIS SO IT LOOKS NICER
\fancypagestyle{titlepage}{
\fancyhead{}
\fancyfoot[CE, CO]{\textcolor{gray}{Via Ramarini 32, Monterotondo RM, 00015 \thinspace\thinspace $\bullet$ \thinspace\thinspace\mobilephonesymbol +39~(388)~059~1507 \thinspace\thinspace $\bullet$ \thinspace\thinspace \faSkype \thinspace\thinspace \href{skype:rabea.jesser}{rabea.jesser} \thinspace\thinspace \thinspace\thinspace \emailsymbol \href{mailto:rabeajesser@web.de}{rabeajesser@web.de} \thinspace\thinspace $\bullet$ \thinspace\thinspace \faLinkedin \thinspace\thinspace \href{https://www.linkedin.com/in/rabeajesser/}{https://www.linkedin.com/in/rabeajesser/}}}}


\name{\vspace{-5mm}Rabea Jesser}{\vspace{3mm}}

\begin{document}
%Breaking down lignose - Gregg Beckham\\
%depolymerize lignin in an integrated biorefinery context and subsequently upgrade lignin-derived aromatics into value-added chemicals to realize a more holistic, integrated approach to biomass conversion\\
%biological upgrading of sugars\\
%bench-scale catalytic and separations processes \\
%enzymatic processes for both deconstructing cellulose and also assembling it \\
%use of novel metabolic enzymes for biologically producing fuel precursors\\
%begin a working group to figure out ways to incorporate concepts of the Circular Materials Economy into the designs of everything from coffee mug lids to wind turbines, cutting down on waste generated by today’s linear economy model. \\



%%%%%%%%%%%%% 			THOMAS NOTES 			%%%%%%%%%%%%%%%%%%%%%%%

%%%%%%%%%%%%%%%%%%%%%%%%%%%%%%%%%%%%%%%%%%%%%%%%%%%%%


\thispagestyle{titlepage}

\recipient{National Bioenergy Center}{15013 Denver West Parkway, Golden, CO 80401}
\vspace*{-8mm}
%%%% DONT FORGET TO UPDATE THIS OR USE \today
\date{\today}
%%%%% IF YOU KNOW WHO YOU'RE SENDING THIS TO THEN GIVE THEIR NAME
\opening{Dear Dr.  Beckham,}
\closing{Sincerely,}
%\enclosure[Note]{M.Sc. Thesis, B.Sc., references and a more complete CV can also be sent upon request.}          
\makelettertitle
\begin{spacing}{1.15}

%Just like yourself I don't wish to see the natural world being degraded beyond repair and am fed up with being a contributor or at least a bystander in this destruction. 
%It is my abiding desire to be part of the solution and contribute to your efforts in providing a holistic, integrated approach to biomass conversion and in finding ways to integrate concepts of Circular Materials Economy into the designs of everything. 
%With my hands-on experience with manipulating and analyzing different microorganisms, the strong work ethic and the team spirit I gained in the iGEM (international Genetically Engineered Machine) competition and the adaptive capacity I showed as a Technical Officer for Europe's flagship laboratory for the life sciences in Rome, I bring the neccessary background knowledge and the enthusiasm and eagerness to learn to make up for any deficiency.\par\vspace*{1mm}
\vspace*{-2mm}
Sharing NREL's passion for renewables, I will wholeheartedly invest my laboratory experience, strong work ethic and analytical thinking to find new biocatalysts and unravel their functionality.
My experience in engineering and analyzing microorganisms like algae, cyanobacteria and \textit{E. coli} combined with the team spirit and interpersonal skills I developed during the iGEM (international Genetically Engineered Machine) competition make me an excellent fit for your group. 
Furthermore, the initiative and ability to adapt to new situations I have displayed while working at Europe's flagship laboratory for the life sciences (EMBL) ensure that I can quickly transition to your lab and begin making contributions from day one.\par\vspace*{1.5mm}

I have always had a special interest in applying molecular biology for the good of the environment, so for my B.Sc. thesis I wanted to explore the potential of diatoms as an alternative expression system for recombinant antibodies. 
Later during my M.Sc. thesis, I looked into ways of using the endogenous CRISPR-Cas system of Synechocystis sp. PCC6803 for metabolic engineering. I indentified inactieve and defective mutants of one of the crRNA maturation endonucleases, opening up ways to employ this enyzme for adjustable regulation
{\href{https://www.tandfonline.com/eprint/vmAQ3vjYGdZIZpIKQTIT/full}{\textcolor{blue}{{\textit{Publication}}}}}.
In the process I became familiar with extracting and analyzing RNA \textit{in vitro} and from Synechocystis cultures, documenting pigment content via photometric measurements and producing genetic constructs and recombinant proteins in \textit{E. coli}.
\par%\vspace*{1.5mm}
 
During this time I also participated in an interdisciplinary team of 20 students to develop a low-cost, multiplexed and label-free diagnostic tool as part of the Univerisity of Freiburg's team in the iGEM competition.
Within eight months of wet lab we produced high-value {\href{https://www.ncbi.nlm.nih.gov/pubmed/29803867}{\textcolor{blue}{\textit{data}}}}, presenting our findings on our {\href{http://2015.igem.org/Team:Freiburg/Home_Intro}{\textcolor{blue}{\textit{website}}}} and at iGEM's international conference at MIT, where we were awarded a Gold medal and nominated for Best Health and Medicine Project, Best Innovation in Medicine and Best Wiki.
Throughout iGEM we successfully ran our own lab independently, including ordering reagents, communicating with companies and funding agencies for financial or material aid, making and sterilizing media and keeping the lab and our data organized.\par%\vspace*{1.5mm}
		
After my M.Sc., I began working in EMBL's Genetic Engineering Facility, which evolved into the Genetic and Viral Engineering Facility soon after I started. 
I quickly adapted and engaged myself in implementing new protocols for the production and quantification of viral vector tools in mammalian cell culture as well as producing a number of genetic constructs via recombineering and other cloning strategies.
Initially our two person team only provided recombinant AAV, however, as my skills progressed, our portfolio has expanded to include recombinant lentivirus and we are in the process of adding HSV.
 In addition to the laboratory work, I have taken the initiative to extend the facility's database and automate calculations for protocols using Excel and FileMaker, helping streamline the lab and more consistently produce and record valuable data.
Additionally, I am continually applying techniques I learned from a professional development course to deliver high quality final products in a timely fashion to our partners in over 12 projects.\par%\vspace*{1.5mm}

Although EMBL offered extensive experience in the lab and provided room for growth, the research focus on neurobiology and epigenetics is simply not after my own heart.
Your lab offers the opportunity both to follow my passion to help make a sustainable and circular economy a reality. 
If you agree, I would appreciate the opportunity to discuss any potential position and my contributions in more detail. 
Thank you for your time and consideration, I look forward to hearing from you.

\end{spacing}
%\vspace*{1mm} 
\makeletterclosing
\end{document}