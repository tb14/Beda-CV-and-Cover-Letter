\documentclass[11pt,a4paper,sans]{moderncv}

\usepackage{float}
\usepackage{caption}
\usepackage{fancyhdr}
\usepackage{setspace}
\usepackage{verbatim}
\usepackage[scale=0.8, top = 1.8cm, bottom = 3cm]{geometry}

\moderncvstyle{casual}                             
\moderncvcolor{blue}                               


%%%% FIX THIS SO IT LOOKS NICER
\fancypagestyle{titlepage}{
\fancyhead{}
\fancyfoot[CE, CO]{\textcolor{gray}{Via Ramarini 32, Monterotondo RM, 00015 \thinspace\thinspace $\bullet$ \thinspace\thinspace\mobilephonesymbol +39~(388)~059~1507 \thinspace\thinspace $\bullet$ \thinspace\thinspace \faSkype \thinspace\thinspace \href{skype:rabea.jesser}{rabea.jesser} \thinspace\thinspace \thinspace\thinspace \emailsymbol \href{mailto:rabeajesser@web.de}{rabeajesser@web.de} \thinspace\thinspace $\bullet$ \thinspace\thinspace \faLinkedin \thinspace\thinspace \href{https://www.linkedin.com/in/rabeajesser/}{https://www.linkedin.com/in/rabeajesser/}}}}


\name{\vspace{-5mm}Rabea Jesser}{\vspace{3mm}}

\begin{document}

%%%%%%%%%%%%%%%%%% RABEA NOTE %%%%%%%%%%%%%%%%%%%%%%


%%%%%%%%%%%%%%%%%%%RESEARCH FOCUS%%%%%%%%%%%%%
%( i) developing larger and more sophisticated molecular toolkits for cyanobacteria and optimizing those tools to ensure high recombineering efficiencies,\\
%ii) using genome-wide engineering techniques to identify genes which increase performance of cyanobacteria under stress conditions (salt, nutrient limitation, high/low light, etc) and\\
%iii) designing production strains capable of using CO2 and sunlight to produce molecules of interest.\\

Colorado School of Mines researcher gets FIVE-year DOE grant to develop predictive metabolic models for algae - that was in JULY 2018:) to develop predictive metabolic models for algae will focus on Chromochloris zofingiensis 
The metabolic model for C. zofingiensis will be developed using a systems biology approach, in hopes of predicting ways to engineer around one of the biggest challenges in growing algae compared to other organisms: changes in light when grown outside.\\
C. zofingiensis: when it switches from photoautotrophic growth to heterotrophic growth , it greatly increases the production of triacylglyrcerols (TAGs) , biodiesel precursors, and astaxanthin, which can be used as a value added nutraceutical.\\
They want to build a metabolic model which will be used to evaluate genetic alternations leading to higher flux towards TAGs and astaxanthin.\\
Organism:  fast-growing, can be cultured at high densities\\

\thispagestyle{titlepage}
\recipient{Colorado School of Mines}{1500 Illinois St., Golden, CO 80401}
%%%% DONT FORGET TO UPDATE THIS OR USE \today
\date{\today}
%%%%% IF YOU KNOW WHO YOU'RE SENDING THIS TO THEN GIVE THEIR NAME
\opening{Dear Dr. Nanette Boyle,}
\closing{Sincerely,}
%\enclosure[Note]{Reference contacts and/or letters of reference will be sent upon request.}          
\makelettertitle
\begin{spacing}{1.1}
Just like yourself I am aware that humanity is in a big need of an alternative of a more sustainable source of energy and am convinced that photosynthetic organisms could play a key role in that. 
My hands-on experience with manipulating and analyzing algae and cyanobacteria, the strong work ethic and the team spirit I gained in the iGEM (international Genetically Engineered Machine) competition and the adaptive capacity I showed as a Technical Officer for Europe's flagship laboratory for the life sciences in Rome, make me a perfect addition to your laboratory. Furthermore, I work with precise attention to detail, am eager to learn and am impassioned for the world of photosynthetic organisms.
\par%\vspace*{1mm}
For my theses {\href{https://www.ncbi.nlm.nih.gov/pubmed/29517395}{\textcolor{blue}{{\textit{(Publication)}}}}} I manipulated and analyzed algae (\textit{P. tricornutum}, \textit{N. ocecanica}) and cyanobacteria (Synechocystis sp. PCC6803). 
More specifically, I used different transformation methods, analyzed their protein and RNA expression and explored the endogenous Synechocystis CRISPR-Cas system for appplication in metabolic engineering. 
The wide array of lectures and courses in plant sciences, translational biology and microbiology has left me with a solid foundation to learn and
develop new procedures and protocols in the lab.
As an example, during a reorganization of my current lab, I consulted relevant literature to establish standard protocols for the production and quantification of viral vector tools in mammalian cell culture.
Here, I precisely documented and monitored the generated data to identify areas for improvement and confered with my supervisor to continually improve the procedure. \par%\vspace*{0.5mm}
For the iGEM competition I participated in an interdisciplinary team of 20 highly motivated students.
Within a month we agreed on an influential topic, and after only eight months of wet lab we produced high-value  {\href{https://www.ncbi.nlm.nih.gov/pubmed/29803867}{\textcolor{blue}{\textit{data}}}}, finally presenting our team's findings on our {\href{http://2015.igem.org/Team:Freiburg}{\textcolor{blue}{\textit{website}}}} and at an international conference. 
This experience has not only left me with a strong work ethic, but also with the ability to efficiently identify the essentials from scientific findings and to present these in a comprehensive way. 
During this time we successfully ran our own lab independently, including ordering reagents, communicating with companies and funding agencies for financial or material aid, making and sterilizing media and keeping the lab and our data organized. \par 

At my current position I not only produce high-quality viral vector tools but also employ different genetic engineering methods in \textit{E. coli} (like P1 transduction and recombineering) and apply various cloning approaches for the generation of genetic constructs.
Furthermore, I took the initiative to improve the facility's database (FileMaker)---leaving me with excellent organizational skills I can invest to ensure accurate data recording and storage in the Boyle lab.
I have also attended a professional course on presenting with impact and project management for scientists and am able to apply these techniques to balance multiple projects---currently more than 12 at once---while managing my time to deliver final products as quickly as possible.  
Additionally, as part of a {\href{https://adamascienza.com/summer-in-science/}{\textcolor{blue}{\textit{summer school}}}} for high school students I supervised and coordinated students alongside my daily schedule and am well prepared to supervise or train fellow team members. \par%\vspace*{1mm}

Given my skill set, I believe I could make valuable contributions to the Boyle lab and it would be an honor to advande the molecular toolkit for cyanobacteria and to identify key genes for the design of cyanobaterial production strains.
If you agree, I would appreciate the opportunity to discuss the position and my skills in more detail. 
Thank you for your time and consideration, I look forward to hearing from you soon.


 

\end{spacing}
%\vspace*{1mm} 
\makeletterclosing
\end{document}