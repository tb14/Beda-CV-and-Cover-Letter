\documentclass[12pt]{article}
\usepackage[top= 1in, bottom= 1in, left= 1in, right= 1in]{geometry}
\usepackage{setspace}
%\usepackage{amssymb}
\begin{document}
\begin{spacing}{1.5}
\begin{centering}
\underline{\huge{\textbf{Tips from NREL Resume Workshop}}}
\end{centering}
\vspace{5mm}

%\renewcommand{\labelenumii}{\Roman{enumii}}
 \begin{enumerate}
   {\LARGE \item \textbf{Top Mistakes}}
   		\begin{enumerate}
    		 	{\large \item \emph{Lack of specifics}}
     				\begin{enumerate}
        				 \item[] Wherever possible you should be specific--- what were the objectives, did you have collaborators, what did you accomplish?
       			\end{enumerate}
       		{\large \item \emph{Highlighting duties, not accomplishments}}
    			 	\begin{enumerate}
         				\item[] Potential employers want to see what you were able to do in your time at your last job---not what was asked of you. Often this means typing the same facts but rewording/changing the focus: ``I was tasked with setting up a new 	database'' $\Rightarrow$ ``I was able to create a highly organized, easily searchable database for more streamlined research''
     				\end{enumerate}
     			{\large \item \emph{1 Size Fits All Approach}}
    				 \begin{enumerate}
         				\item[] This is a no-brainer: you should craft a new Resume/Cover letter for each application
      				 \end{enumerate}
     			{\large \item \emph{Too Few Action Verbs}}
     				\begin{enumerate}
        				 \item[] Try to avoid the passive voice, you want them to imagine you doing lots of great stuff in their lab, taking charge, etc.
      				 \end{enumerate}
     			{\large \item \emph{Too Visually ``Busy''}}
     				\begin{enumerate}
				         \item[] Whitespace can actually be a good thing: if your resume is just one giant block of text, it will overwhelm them
			       \end{enumerate}
 
 		\end{enumerate}
   {\LARGE \item \textbf{Preparing for the Application and Beyond}}
    		 	\begin{enumerate}
    		 	{\large \item \emph{What You Should Know}}
     				\begin{enumerate}
        				 \item The job
        				 \begin{enumerate}
         					\item[] Obviously you're going to need to know about the specifics of your job posting: duties, minimum requirements, etc.
         					\end{enumerate}
        				 \item The company
        				 \begin{enumerate}
         					\item[] You should know what the company does, how many people work there, what their mission statement is, how they see themselves, how they accomplish goals, what successes they've had.
         					\end{enumerate}
        				 \item The latest in the field
        				 \begin{enumerate}
         					\item[] Try to keep abreast of what is going on in the field (especially for the interview, and doubly especially if the company is a leader in the field): if there's been a big breakthrough, be able to talk a bit about it.
         					\end{enumerate}
        				 \item The bosses
        				 \begin{enumerate}
         					\item[] Know what your potential bosses are working on. I think the only reason I got the job at NREL is that I had read every paper my boss had published in the past 5 years and could hold a conversation on it.
         					\end{enumerate}
       			\end{enumerate}
    			 {\large \item \emph{Have questions ready for the end of the interview}}
    				\begin{enumerate}
        				 \item What does an average day look like for me there?
        				 \item Who would I be reporting to? Would I have multiple officers above me?
        				 \item What is the company culture like?
        				 \item What is the area like?
       			\end{enumerate}
     			{\large \item \emph{LinkedIn}}
    				 \begin{enumerate}
         				\item Keep it up-to-date! 
         					\begin{enumerate}
         					\item[] They talked about this a lot so be sure that your profile is up-to-date, looks good and is accessible!
         					\end{enumerate}
         				\item Other Social media
         					\begin{enumerate}
         					\item[] You should also be aware that they might look at your other social media as well \dots
         					\end{enumerate}
      				 \end{enumerate}
     			{\large \item \emph{Craft your message for your audience}}
     				\begin{enumerate}
        				 \item[] You might find yourself talking to an HR person rather than someone in the lab (at least for the first round of interviews) so be sure you can talk about your work both on an expert level and to the average person who knows what the lab-people are looking for, but may not have a Ph.D. in Microbiology
      				 \end{enumerate}
     			{\large \item \emph{Personal Interests vs Volunteer Experience}}
     				\begin{enumerate}
				         \item[] On your actual application, you should know there is a difference between a personal interest (``I like hiking'') and an experience (``I spent 2 summers volunteering leading trips into the wilderness''). They will be interested in your volunteer experiences but will not much care for the fact you like ``Kultur, Natur und Sport''.
			       \end{enumerate}
		       \end{enumerate}
		{\LARGE \item \textbf{Specific to NREL}}
     			\begin{enumerate}
			        {\large \item \emph{Include Your Publications and aim for a CV}}
			         \begin{enumerate}
         					\item[] They said to absolutely mention all your publications and that they would prefer a CV (which has a lot more information).
 						\vspace{2mm}
         					\item[] They also mentioned you should not absolutely feel that you have to limit yourself to one page!
          				\end{enumerate}
			         
		       \end{enumerate}
 


 \end{enumerate}

\end{spacing}
\end{document}

%\begin{enumerate}
%\item The first item
%\begin{enumerate}
%\item Nested item 1
%\item Nested item 2
%\end{enumerate}
%\item The second item
%\item The third etc \ldots
%\end{enumerate}