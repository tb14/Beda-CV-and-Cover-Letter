\documentclass[11pt,a4paper,sans]{moderncv}

\usepackage{float}
\usepackage{caption}
\usepackage{fancyhdr}
\usepackage{setspace}
\usepackage{verbatim}
\usepackage[scale=0.8, top = 1.8cm, bottom = 3cm]{geometry}

\moderncvstyle{casual}                             
\moderncvcolor{blue}                               


%%%% FIX THIS SO IT LOOKS NICER
\fancypagestyle{titlepage}{
\fancyhead{}
\fancyfoot[CE, CO]{\textcolor{gray}{Via Ramarini 32, Monterotondo RM, 00015 \thinspace\thinspace $\bullet$ \thinspace\thinspace\mobilephonesymbol +39~(388)~059~1507 \thinspace\thinspace $\bullet$ \thinspace\thinspace \faSkype \thinspace\thinspace \href{skype:rabea.jesser}{rabea.jesser} \thinspace\thinspace \thinspace\thinspace \emailsymbol \href{mailto:rabeajesser@web.de}{rabeajesser@web.de} \thinspace\thinspace $\bullet$ \thinspace\thinspace \faLinkedin \thinspace\thinspace \href{https://www.linkedin.com/in/rabeajesser/}{https://www.linkedin.com/in/rabeajesser/}}}}


\name{\vspace{-5mm}Rabea Jesser}{\vspace{3mm}}

\begin{document}

%%%%%%%%%%%%%%%%%% THOMAS NOTE %%%%%%%%%%%%%%%%%%%%%
%
% My two main points:
%
%		1) I would keep the interesting stuff about your M.Sc. in the second paragraph and move the iGEM stuff to the third or combining them into one paragraph with the M.Sc. stuff upfront and the iGEM after,
%			this would be a good way to `sandwich' your soft sills between your hard skills. Just a thought
%
%		2) Always, always, always try and draw what your work has done to what they are doing. 
%

%%%%%%%%%%%%%%%%%% RABEA NOTE %%%%%%%%%%%%%%%%%%%%%%


%%%%%%%%%%%%%%%%%%%RESEARCH FOCUS%%%%%%%%%%%%%
%( i) developing larger and more sophisticated molecular toolkits for cyanobacteria and optimizing those tools to ensure high recombineering efficiencies,\\
%ii) using genome-wide engineering techniques to identify genes which increase performance of cyanobacteria under stress conditions (salt, nutrient limitation, high/low light, etc) and\\
%iii) designing production strains capable of using CO2 and sunlight to produce molecules of interest.\\

%%%%%%%%%%%%%%%%%University of Colorado Genetic Engineering%%%%%%%%%%%
%%%%%%%%%%
%Ryan Gill is a Professor in the Department of Chemical and Biological Engineering at CU Boulder and is the Associate Director for Research at RASEI. He was the founding Managing Director at the Colorado Center for Biorefining and Biofuels (C2B2) and was co-founder and board member at OPXBio.%%%%%%%%%%
%%%%RESEARCH INTEREST
%%%Directed Genome Evolution: New Tools and Applications 
%Our research falls within the general fields of METABOLIC ENGINEERING, SYNTHETIC BIOLOGY, DIRECTED EVOLUTION, and GENOMICS and is targeted primarily towards the development biorefining processes for the SUSTAINABLE PRODUCTION OF FUELS, COMMODITY CHEMICALS, PHARMACEUTICALS. We are generally focused on the development of new i) tools for STRAIN ENGINEERING (i.e. genome engineering), ii) methods for efficiently DETERMINING the GENETIC BASIS of so engineered strains (i.e. functional genomics), and iii) frameworks for rationalizing RELATIONSHIPS BETWEEN GENOME STRUCTURE AND FUNCTION. Our specific projects include:

% Creation, development, and demonstration of novel GENOME-ENGINEERING TOOLS.
% Metabolic engineering for SUSTAINABLE and FUNGIBLE BIO-DIESEL
% Metabolic engineering for sustainable and fungible BIO-GASOLINE production
% Creation, development, and demonstration of new approaches for ANTIMICROBIAL DISCOVERY
% Genome-engineering to IMPROVE CELLULOSIC biofuels production

%Our research is supported through a number of sources including current or previous grants from the NSF, NIH, DOE, State of Colorado, University of Colorado, Colorado Center for Biorefining and Biofuels, the Cystic Fibrosis Foundation, Agilent, Shell, Dupont, and Opx Bioproducts.%%%%%%%%%%%%%%
%%%%Involved here: 
%%%%Design and engineering of synthetic control architectures

%%advance biological engineering to a new level of throughput and complexity. Guided by computer-aided design platforms, this research will develop new finely tuned microbial strains containing orthogonal regulatory networks. The redesigned strains will be tailored to grow on a variety of feedstocks, tolerate toxic metabolites and other stresses, and produce valuable molecules. Building on technologies and knowledge developed with prior DOE funding from the Biosystems Design program, the project will further develop the necessary technical and computational infrastructure to achieve its goals, initially working with Escherichia coli and Saccharomyces cerevisiae and later transferring the technologies to other non-model bacteria and yeast species.
%publications:
%%%%https://www.ncbi.nlm.nih.gov/pubmed/25856528 
%%%%https://pubs.acs.org/doi/10.1021/acssynbio.5b00009

%%%%%%Notes on how to write%%%%%%%%
%%%%Your abilities + the companys needs = desirable results
%%%%don't use I think or something similar
%%%%use this:    Excellent,    Great,    Terrific,    Strong,  Outstanding,    Unique instead of perfect or best

\thispagestyle{titlepage}
\recipient{Sustainability, Energy and Environment Complex}{4001 Discovery Drive, Boulder, CO 80303}
%%%% DONT FORGET TO UPDATE THIS OR USE \today
\date{\today}
%%%%% IF YOU KNOW WHO YOU'RE SENDING THIS TO THEN GIVE THEIR NAME
\opening{Dear Dr. Gill,}
\closing{Sincerely,}
%\enclosure[Note]{Reference contacts and/or letters of reference will be sent upon request.}          
\makelettertitle
\begin{spacing}{1.1}

After being introduced to the potentials of recombineering by Dr. Jim Sawitzke and having had the opportunity to apply recobineering in some projects, I am ready to expand on my hands-on experience and help engineer new strains in your lab. % 
My experience in engineering and analyzing microorganisms like algae, cyanobacteria and \textit{E. coli} combined with my understanding of CRISPR/Cas systems in general and the Synechocystis CRISPR/Cas system in particular, has prepared me to test new genetic tools in a variety of organisms.
The team spirit and interpersonal skills I developed during the iGEM (international Genetically Engineered Machine) competition, and the initiative and ability to adapt to new situations I have displayed while working for Dr. Jim Sawitzke ensure that I can quickly transition to your lab and begin making contributions from day one.\par%\vspace*{3mm}

For my M.Sc. thesis I explored the use of the endogenous CRISPR-Cas system of Synechocystis sp. PCC6803 in metabolic engineering and thereby further characterized the Synechocystis crRNA maturation endonuclease Cas6-1
{\href{https://www.tandfonline.com/eprint/vmAQ3vjYGdZIZpIKQTIT/full}{\textcolor{blue}{{\textit{Publication}}}}}.
This experience has prepared me to analyze, translate and reengineer the functionality of CRISPR/Cas systems for other purposes. 
During my M.Sc. I participated in an interdisciplinary team of 20 students to develop a low-cost, multiplexed and label-free diagnostic tool in the iGEM competition.
Within eight months of wet lab we produced high-value {\href{https://www.ncbi.nlm.nih.gov/pubmed/29803867}{\textcolor{blue}{\textit{data}}}}, presenting our findings on our 
{\href{http://2015.igem.org/Team:Freiburg/Home_Intro}{\textcolor{blue}{\textit{website}}}} and at iGEM's international conference at MIT, where we were awarded a Gold medal and nominated for Best Health and Medicine Project, Best Innovation in Medicine and Best Wiki.
Throughout iGEM we successfully ran our own lab independently, including ordering, preparing and sterilizing reagents, communicating with companies and funding agencies and keeping the lab and our data organized. \par

After my M.Sc. I began working for Dr. Jim Sawitzke in the Genetic Engineering Facility of at Europe's flagship laboratory for the life sciences (EMBL). There I gained hands-on experience with oligo recombineering and fragment deltion on a plasmid using the LambdaRed system and { in vivo} cloning using the RecET system.
Soon after I started, the facility evolved into being the Genetic and Viral Engineering Facility. I quickly adapted and engaged myself in implementing new protocols for the production and quantification of viral vector tools in mammalian cell culture. 	
Initially our two person team only provided recombinant AAV, however, as my skills progressed, our portfolio has expanded to include recombinant lentivirus and we are in the process of adding HSV.
In addition to the laboratory work, I have taken the initiative to extend the facility's database and automate calculations for protocols using Excel and Filemaker, helping streamline the lab and more consistently produce and record valuable data. 
Furthermore, I am continually applying techniques I learned from a professional development course on project management for scientists across more than 12 projects to deliver high quality final products in a timely fashion to our partners.\par%\vspace*{3mm}

Although EMBL offered extensive experience in the lab and provided room for growth, the research focus on neurobiology and epigenetics is simply not after my own heart and I would rather like to use my knowledge about recombineering and CRISPR/Cas systems towards developing the next generation of clean fuels and chemicals.
If you agree, I would appreciate the opportunity to discuss any potential position and my contributions in more detail. 
Thank you for your time and consideration, I look forward to hearing from you.
 

 

\end{spacing}
%\vspace*{1mm} 
\makeletterclosing
\end{document}